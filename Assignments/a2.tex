\documentclass[12pt,twoside]{article}

%\usepackage{graphicx}
\usepackage{enumitem}
\usepackage{mathtools}
\usepackage[svgnames]{xcolor}
\usepackage{afterpage}
\usepackage[none]{hyphenat}
\usepackage[document]{ragged2e}

\usepackage{geometry}
 %\geometry{
 %letterpaper,
 %textwidth=7in
 %}
\usepackage[utf8]{inputenc}

%\usepackage[cmintegrals,cmbraces,vvarbb]{newtxmath}
\usepackage{fouriernc}
\usepackage[T1]{fontenc}
%\usepackage{palatino}
\usepackage{microtype}

\usepackage{fancyhdr}
\setlength{\headheight}{15pt}% ...at least 15

\usepackage[colorlinks=true, citecolor=Grey,
            linkcolor=DarkOrchid, urlcolor=ForestGreen]{hyperref}


\pagestyle{fancy}
\fancyhf{}
\fancyhead[RE,LO]{\textcolor{BlueViolet}{Gravitational Waves}}
\fancyhead[LE,RO]{\textcolor{ForestGreen}{ph237: 2020}}

\fancyfoot[RE,LO]{\textcolor{Orange}{Caltech}}
\fancyfoot[LE,RO]{\textcolor{Orange}{Physics, Math, and Astronomy}}

%\input mydefs.tex
\def\vev#1{\left\langle #1\right\rangle}
\def\hb{\hfill\break}
\definecolor{awesome}{rgb}{1.0, 0.13, 0.32}


\begin{document}
%
\centerline{\large\bf  \hfill Assignment II \hfill  \today}

\medskip
\begin{description}
\item[{\bf Reading:}] Lecture notes. \\
\item[{\bf Problems:} \hfill ] Due April 23$^{\rm th}$.
\end{description}


\medskip

\begin{enumerate}

\item
{\bf Interferometer Antenna Pattern}
\begin{figure}[h]
    \centering
    \includegraphics[width=12cm]{Figures/A5_1a.pdf}

    \label{fig:A5_1a}
\end{figure}
\begin{itemize}

\item[\bf a)] In the long wavelength approximation, compute the antenna response $F_+$ and $F_\times$ for a LIGO-like interferometer for waves incident from the $\theta$, $\phi$ direction. Assume that the interferometer
is just a simple Michelson interferometer with 4\,km long arms. The two arms are orthogonal to each other.

Make a cool 3D plot for each polarization \\
(e.g. \url{http://www.antenna-theory.com/antennas/dipole.php}) \\
and also a couple of 2D slices \\
(e.g. \url{http://hfradio.org/ace-hf/ace-hf-antenna_is_key.html}).

\item[\bf b)] Using the quadrupole formula, compute the emissivity of the LIGO detector toward that direction:
wave amplitude $h_+$ and $h_x$ for waves generated toward that direction when the differential arm length is oscillating at an angular frequency, $\omega$, and amplitude, $\delta L$.

What is the relationship between the sensitivity pattern and the emission pattern?


\item[\bf c)] In part b), what fraction of the kinetic energy of the oscillating masses is converted into gravitational radiation?


\end{itemize}

\clearpage
\item
{\bf Binary Stars} \\
Gravitational Radiation from binary stars.
\begin{itemize}

\item[\bf a)] The binary star system $\iota$ Boo is in the Tian Qiang
  constellation at a distance of 12 pc from the Earth. Assume that each star has a mass of 1 $M_{\odot}$ and that the system has an orbital period of 6.5\,hours. \\
  Compute the frequency and amplitude of the strain as measured
  at the earth. For the purposes of this calculation, assume a
  ``face-on'' inclination angle of the binary; i.e. its orbital
  angular momentum vector points directly at the earth.

\item[\bf b)] Assume that the total energy of the binary stars can
  be well estimated by Newtonian mechanics (since they are from each
  other and the velocities are non-relativistic). Compute the
  gravitational wave luminosity, $L_{GW}$ as a function of orbital
  radius $R$. Using this result, compute and plot the time evoultion
  of $R$, assuming that the energy lost per orbit is small.

\item[\bf c)] Now compare the intensity of the radiation (in W/m$^2$)
as measured on the Earth with that of the sun ($\sim 1\,\mbox{kW/m}^2$).

\end{itemize}



\end{enumerate}

\bigskip
{\color{awesome} \hrule}
\end{document}
