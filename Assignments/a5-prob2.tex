\item \textbf{Radiation Pressure Fluctuations:} \\

  Given the Adjacency matrix approach to solving for the steady-state fields in the interferometer, we can now find the DC (static) fields everywhere, as well as the steady-state fields given some small sinusoidal modulation.

  %In this problem, we want to expand that technique to include the effects of a fluctuating radiation pressure force.

  In general, phase modulation (PM) and amplitude modulation (AM) can be mixed into some linear admixture of both, and so we want a new Adjacency matrix that incorporates this mixing capability.

  \begin{enumerate}

  \item For PM, we begin with the expression $E = E_0 e^{i \omega_0 t + i \Gamma cos(\omega t)}$, and expand to first order to find the amplitude and phase of the two first order sidebands (see lecture notes and/or the Jacobi-Anger expansion). Do the same for AM, where $E = E_0 (1 + \Gamma cos(\omega t)) e^{i \omega_0 t}$. How do these compare to the PM sidebands?

  \item Thinking about a Fabry-Perot optical cavity (with suspended mirrors), what are some mechanisms by which AM can be converted into PM and vice versa? No calculations necessary -- just use your physical intuition.

  \item Make a \emph{phasor} diagram of the carrier + sidebands including the 4 following cases (at time $t = 0$):
     \begin{enumerate}
     \item phase modulation (cosine)
     \item phase modulation (sine)
     \item amplitude modulation (cosine)
     \item ampltiude modulation (sine)
     \end{enumerate}
  Assuming that we are in the frame co-rotating with the carrier, what can you say about the contribution of the sidebands to the amplitude or phase of the carrier? This can be qualitative.










  \end{enumerate}
