\documentclass[12pt,twoside]{article}

%\usepackage{graphicx}
\usepackage{enumitem}
\usepackage{mathtools}
\usepackage[svgnames]{xcolor}
\usepackage{afterpage}
\usepackage[none]{hyphenat}
\usepackage[document]{ragged2e}

\usepackage{geometry}
 %\geometry{
 %letterpaper,
 %textwidth=7in
 %}
\usepackage[utf8]{inputenc}

%\usepackage[cmintegrals,cmbraces,vvarbb]{newtxmath}
\usepackage{fouriernc}
\usepackage[T1]{fontenc}
%\usepackage{palatino}
\usepackage{microtype}

\usepackage{fancyhdr}
\setlength{\headheight}{15pt}% ...at least 15

\usepackage[colorlinks=true, citecolor=Grey,
            linkcolor=DarkOrchid, urlcolor=ForestGreen]{hyperref}


\pagestyle{fancy}
\fancyhf{}
\fancyhead[RE,LO]{\textcolor{BlueViolet}{Gravitational Waves}}
\fancyhead[LE,RO]{\textcolor{ForestGreen}{ph237: 2020}}

\fancyfoot[RE,LO]{\textcolor{Orange}{Caltech}}
\fancyfoot[LE,RO]{\textcolor{Orange}{Physics, Math, and Astronomy}}

%\input mydefs.tex
\def\vev#1{\left\langle #1\right\rangle}
\def\hb{\hfill\break}
\definecolor{awesome}{rgb}{1.0, 0.13, 0.32}

\begin{document}
%
\centerline{\large\bf  \hfill Assignment III \hfill  \today}

\medskip
\begin{description}
\item[{\bf Reading:}] Lecture notes. \\
\item[{\bf Problems:} \hfill ] Due April 30$^{\rm th}$.
\end{description}


\medskip

\begin{enumerate}

\item  {\bf  Antenna Pattern of LIGO at high frequencies}
\begin{enumerate}
\item For two arms with length, $L$, along $x$ and $y$ axis,
  compute the antenna pattern as a function of frequency.
  More specifically, for gravitational waves propagation along the
  $(\theta,\phi)$ direction, and with $+,\times$ polarizations, with  frequency $\Omega$, compute the phase shift difference $\delta\phi$ for light  sent to the two two arms at $t - 2 L/c$ and returning at $t$.
  [Fourier transform $\delta\phi$ to obtain an antenna pattern in frequency domain.]

\item In E\&M, it is often said that sensitivity of an antenna to a particular mode of radiation is related to its emissivity in that mode.
  Is this true for GWs?  

  Suppose we have a ``scalar EM'' field (just to simplify the calculations) wiith an amplitude, $A$, in both arms, and we apply a differential amplitude modulation, $\pm \delta a(t)$ to the arms.
  Let's compute the amplitude of gravitational wave that is emitted along the $(\theta,\phi)$ direction.  

  Unfortunately, we cannot use the quadrupole formula in this case:
  for high frequencies, the wavelength of the radiation can be smaller than the size of the interferometer.

  Show that the following is true:
  the stress energy tensor of the ``scalar EM'' wave, averaged over the optical wavelength (i.e. for frequencies much less than the laser frequency of $\sim10^{15}$\,Hz), is given by
\begin{align}
T_{xx} (t,x,y,z) &\propto  ~~2 A [a(t-x) +a(t-2L+x)] \delta(y) \delta(z) \\
T_{yy} (t,x,y,z) &\propto -2 A [a(t-y) +a(t-2L+y)] \delta(x) \delta(z) 
\end{align}
This is the stress content of the wave field, which exist along the $x$ arm and the $y$ arm.  What is the physical meaning of this stress in LIGO?
\item 
Use the linearized-gravity formula of
\begin{equation}
\square \bar h^{\alpha\beta} =-16 T^{\alpha\beta}
\end{equation} 
to compute $h_{+,\times}$ along the $(\theta,\phi)$ directions.   \textcolor{red}{Hint: starting from
\begin{equation}
\bar h^{\alpha\beta}(t,\mathbf{x}) =4\int d^3\mathbf{x}' \frac{T^{\alpha\beta}(t -|\mathbf{x}-\mathbf{x}'|,\mathbf{x'})}{|\mathbf{x}-\mathbf{x}'|}
\end{equation}
assuming that distance to the source is very far, we can write
\begin{equation}
\bar h^{\alpha\beta}(t,\mathbf{x}) =\frac{4}{R}\int d^3\mathbf{x}' {T^{\alpha\beta}(t -R +\mathbf{N}\cdot\mathbf{x}',\mathbf{x'})}
\end{equation}
where $\mathbf{N}$ is the wave propagation direction.  This integral can be performed analytically. 
}
\item The stress-energy tensor also has a $T_{00}$ component, which describes the energy content of the scalar EM field.  Why did we not account for this?  (What happens to the wave associated with this component, when we go into the TT gauge?) 
\item For a true Maxwell field,  is the Maxwell stress the same as given by part (b)?
\end{enumerate}

\clearpage
\item  {\bf  Signaling Recycling Cavity Optimization}\\
Fig.~\ref{fig:A6_2} shows a simplified diagram of the optical layout of LIGO. The picture shows the power recycling mirror (PRM) and the signal recycling mirror (SRM) in the diagram. In this question, we are going to understand how the signal recycling mirror affects the frequency response of the interferometer and optimize the frequency response for the detection of the binary systems.
\begin{figure}[ht]
\centering
    \includegraphics[width=\columnwidth]{Figures/A6_2.png}
    \caption{Simplified LIGO optical layout\protect  \footnotemark[1].}
    \label{fig:A6_2}

\end{figure}
    \footnotetext[1]{Source: \protect  \url{http://www.physics.gla.ac.uk/igr/index.php?L1=research&L2=signal_recycling}}
\begin{enumerate}
\item Using the simplified layout we mentioned in class where the interferometer is considered as a 3-mirror cavity (shown in Fig.~\ref{fig:A6_2a}), calculate the frequency response of the system using the graph / scattering matrix approach.

  Make a Bode plot of the response in units of W/m from 1\,--\,$10^5$\,Hz.

  The transmission of the input mirror (ITM) and end mirror (ETM) are given. Also, we can microscopically shift the SRM from resonance so as to produce a detuning phase shift $\phi_{SRC}$. The transmission of SRM, $T_{SRM}$ and the detuning phase $\phi_{SRC}$ are two parameters used to characterize the signal recycling cavity. We assume the losses of the mirrors are negligible.

Plot the frequency response for several values of $T_{SRM}$ and $\phi_{SRC}$.
  
    \begin{figure}[ht]
    \centering
    \includegraphics[width=0.5\columnwidth]{Figures/A6_2a.pdf}
    \caption{Three-mirror model of the LIGO interferometer.}
    \label{fig:A6_2a}
    \end{figure}

\end{enumerate}
\end{enumerate}
\bigskip
{\color{awesome} \hrule}
\end{document}
