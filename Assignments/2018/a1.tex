\documentclass[11pt]{article}
%\setlength{\oddsidemargin}{0in}
%\setlength{\topmargin}{0.0in}
%\setlength{\textwidth}{6.7in}
%\setlength{\textheight}{8.5in}
%\usepackage{graphicx}
\usepackage{enumitem}
\usepackage{mathtools}
\usepackage[usenames,dvipsnames]{xcolor}

\usepackage{geometry}
 \geometry{
 letterpaper,
 textwidth=6.8in
 }
\usepackage[utf8]{inputenc}
%\usepackage{libertine}
%\usepackage{libertinust1math}
\usepackage[libertine,cmintegrals,cmbraces,vvarbb]{newtxmath}
%\usepackage{newtxmath}
%\usepackage[osf]{ebgaramond}
\usepackage[T1]{fontenc}
%\usepackage{palatino}
\usepackage{microtype}

\usepackage{fancyhdr}
\usepackage[colorlinks=true]{hyperref}

\pagestyle{fancy}
\fancyhf{}
\fancyhead[R,LO]{\textcolor{BlueViolet}{Gravitational Waves}}
\fancyhead[L,RO]{ph237: 2018}

\fancyfoot[R,LO]{\textcolor{Orange}{Caltech}}
\fancyfoot[L,RO]{\textcolor{Orange}{Physics, Math, and Astronomy}}

%\input mydefs.tex
\def\vev#1{\left\langle #1\right\rangle}
\def\hb{\hfill\break}

\begin{document}
%
\centerline{\large\bf  \hfill Assignment I \hfill  4/8/18}

\medskip
\begin{description}
\item[{\bf Reading:}] Lecture Notes. \\
\item[{\bf Problems:} \hfill ] Due April 17, before class.
\end{description}


\medskip

\begin{enumerate}

\item
{\bf Fabry-Perot Cavity} \\
There exists a Fabry-Perot cavity with a length of 4000\,m. The \emph{power} transmission, $T_I$, of the input mirror is 1\% and the transmission of the end mirror, $T_E$, is 10\,ppm. The cavity is illuminated from the input side with a 100\,W laser having a wavelength of 437\,nm.
\begin{itemize}

\item[\bf a)] Using the steady state fields approach, solve for the transmitted \emph{power} as a function of cavity length.

\item[\bf b)] Draw a diagram of the cavity, label each of the nodes, and write down the Adjacency Matrix, $A$, for the Fabry-Perot cavity. Solve for the System Matrix, $G$ using Mathematica or Python or otherwise, and show how one of the elements of $G$ can be used to find the solution in part a).
\end{itemize}

\item
{\bf Frequency Response} \\
In this problem we will compute the frequency response of a LIGO-like interferometer to gravitational waves. Assume that the results from above still hold.
\begin{itemize}

\item[\bf a)] Assume that we now drive the end mirror with a sinusoidal modulation having an amplitude of $x_0$ and a frequency, $\omega$. Write down an expression for the fields reflected from the mirror, utilizing the Jacobi-Anger expansion.

\item[\bf b)] Use your knowledge of the \emph{frequency dependent} System Matrix derived above to compute an expression for the transmitted power. Make a plot of the \emph{transfer function} of the transmitted power as a function of modulation frequency (the y-axis should be in units of Watts/meter).
\textbf{Hint:} Remember that the transmitted field will be the sum of the DC fields (computed above) and the AC fields.

\item[\bf b)] Now write down a larger Adjacency Matrix which represents a Michelson interferometer with Fabry-Perot cavities in place of the usual end mirrors. Assume that there is a small asymmetry in the Michelson, such that the distance from the beamsplitter to one of the FP cavities is 100\,pm larger than the distance to the other cavity.
Make a Bode plot of the \emph{transfer function} as in part b), but instead of the transmission of the FP cavity, use the anti-symmetric (detection) port as the readout.
\end{itemize}

\end{enumerate}

\bigskip
{\color{Sepia} \hrule}
\end{document}
